\documentclass[a4paper,10pt]{article}
\usepackage[utf8]{inputenc}
\usepackage{polski}
\usepackage{graphicx}
\usepackage{listings}
\usepackage[usenames,dvipsnames]{color}
\addtolength{\hoffset}{-1cm}
\addtolength{\voffset}{-2cm}
\addtolength{\textwidth}{2cm}
\addtolength{\textheight}{3cm}
\usepackage{setspace}
\usepackage{indentfirst}
\usepackage{graphicx}
\lstset{
    language=Matlab,
    basicstyle=\scriptsize,
    aboveskip={1.5\baselineskip},
    columns=fixed,
    showstringspaces=false,
    extendedchars=true,
    breaklines=true,
    tabsize=4,
    prebreak = \raisebox{0ex}[0ex][0ex]{\ensuremath{\hookleftarrow}},
    frame=single,
    showtabs=false,
    showspaces=false,
    showstringspaces=false,
    identifierstyle=\ttfamily,
    keywordstyle=\color[rgb]{0,0,1},
    commentstyle=\color[rgb]{0.133,0.545,0.133},
    stringstyle=\color[rgb]{0.627,0.126,0.941},
    numbers=left,
    numberstyle=\tiny,
    stepnumber=1,
    numbersep=5pt,
    captionpos=b,
    escapeinside={\%*}{*)}
}

\def\figurename{Rys.}
\def\lstlistingname{Fun.}

\title{Informatyczne Systemy Sterowania \\ \large Ćwiczenie 4: Sterowanie ekstremalne}

\author{Adam Jordanek 168139, Tomasz Klimek 168092}

\begin{document}
\maketitle

\section{Wstęp}\label{sec:wstęp}
\subsection{Cel ćwiczenia}
%TODO!!!!! SKOPIOWANE Z LISTY
Celem  ćwiczenia jest symulacja działania systemu sterowania ekstremalnego. Sterowaniem 
ekstremalnym nazywamy zadanie sterowania polegające na doprowadzeniu poprzez odpowiednią
zmianę wielkości sterujących do ekstremalnej wartości wielkości sterowanej (lub ekstremalnych 
wartości wielkości sterowanych w przypadku obiektu wielowyjściowego)
\subsection{Plan badań} 
\begin{enumerate}
	\item Symulacja systemu sterowania ekstremalnego - Algorytm 1
	
	\item Symulacja systemu sterowania ekstremalnego - Algorytm 2
	
	\item Symulacja systemu sterowania ekstremalnego - Algorytm 3
	
\end{enumerate}

\subsection{Podział zadań. } 
\begin{enumerate}
		\item Jordanek Adam - 
		\item Klimek Tomasz -
\end{enumerate}

\newpage
\section{Realizacja planu i wyniki}

\section{Wnioski.}\label{sec:wnioski}

\end{document}