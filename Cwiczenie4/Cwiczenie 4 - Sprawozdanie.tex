\documentclass[a4paper,10pt]{article}
\usepackage[utf8]{inputenc}
\usepackage{polski}
\usepackage{graphicx}
\usepackage{listings}
\usepackage[usenames,dvipsnames]{color}
\addtolength{\hoffset}{-1cm}
\addtolength{\voffset}{-2cm}
\addtolength{\textwidth}{2cm}
\addtolength{\textheight}{3cm}
\usepackage{setspace}
\usepackage{indentfirst}
\usepackage{graphicx}
\lstset{
    language=Matlab,
    basicstyle=\scriptsize,
    aboveskip={1.5\baselineskip},
    columns=fixed,
    showstringspaces=false,
    extendedchars=true,
    breaklines=true,
    tabsize=4,
    prebreak = \raisebox{0ex}[0ex][0ex]{\ensuremath{\hookleftarrow}},
    frame=single,
    showtabs=false,
    showspaces=false,
    showstringspaces=false,
    identifierstyle=\ttfamily,
    keywordstyle=\color[rgb]{0,0,1},
    commentstyle=\color[rgb]{0.133,0.545,0.133},
    stringstyle=\color[rgb]{0.627,0.126,0.941},
    numbers=left,
    numberstyle=\tiny,
    stepnumber=1,
    numbersep=5pt,
    captionpos=b,
    escapeinside={\%*}{*)}
}

\def\figurename{Rys.}
\def\lstlistingname{Fun.}

\title{Informatyczne Systemy Sterowania \\ \large Ćwiczenie 4: Sterowanie ekstremalne}

\author{Adam Jordanek 168139, Tomasz Klimek 168092}

\begin{document}
\maketitle

\section{Wstęp}\label{sec:wstęp}
\subsection{Cel ćwiczenia}
%TODO!!!!! SKOPIOWANE Z LISTY
Celem  ćwiczenia jest symulacja działania systemu sterowania ekstremalnego. Sterowaniem 
ekstremalnym nazywamy zadanie sterowania polegające na doprowadzeniu poprzez odpowiednią
zmianę wielkości sterujących do ekstremalnej wartości wielkości sterowanej (lub ekstremalnych 
wartości wielkości sterowanych w przypadku obiektu wielowyjściowego)

\subsection{Plan badań} 
\begin{enumerate}
	\item Symulacja systemu sterowania ekstremalnego - Algorytm 1
	
	\item Symulacja systemu sterowania ekstremalnego - Algorytm 2
	
	\item Symulacja systemu sterowania ekstremalnego - Algorytm 3
	
\end{enumerate}

\subsection{Podział zadań. } 
\begin{enumerate}
		\item Jordanek Adam - 
		\item Klimek Tomasz -
\end{enumerate}

\newpage
\section{Realizacja planu i wyniki}
W zadaniu sterować będziemy dwa obiekty przedstawione przy pomocy poniższych równań.
\begin{eqnarray}
	y = F(u, a, b, c) = (u^{(1)} - a)^2 + (u^{(2)} - b)^2 + c\\
	y = F(u, a, b, c, A, B, \omega_1, \omega_2, t) = (u^{(1)} - (a + Asin(\omega_1t)))^2 + (u^{(2)} - (b + Bsin(\omega_2t)))^2 + c
\end{eqnarray}
Podczas ćwiczenia przyjęliśmy następujące wartości parametrów. \\
$a=1, b=2, c=3, A=2, B=4, \omega_1=0.5, \omega_2=0.25$
%--------------------------------------------------------------------------------------------------------------------------------
%ZADANIE 1
%--------------------------------------------------------------------------------------------------------------------------------
\subsection{Symulacja systemu sterowania ekstremalnego – wersja 1. algorytmu.}

\subsubsection{Obiekt (1)}
W pierwszym zadaniu do wyznaczenia ekstremów posłuży nam poniższa funkcja.
\begin{eqnarray}
	u_n^{(i)} = u_{n-1}^{(i)}-kd_n^{(i)}\\
	d_n^{(i)} = {F(u_{n-1}) - F(u_{n-2}) \over u_{n-1}^{(i)} - u_{n-2}^{(i)}}
\end{eqnarray}

Jednak dla ułatwienia ćwiczenia powyższy algorytm zmodyfikujemy do poniższej postaci.
\begin{eqnarray}
	u_n^{(i)} = u_{n-1}^{(i)}-kd_n\\
	d_n = {F([\begin{array}{ll} u_{n-1}^{(1)} + \Delta u & u_{n-1}^{(2)} + \Delta u\end{array}]) - F(u_{n-1}) \over \Delta u}
\end{eqnarray}

Następnie testować będziemy wpływ parametru $k$ na przebieg sterowania przy użyciu odpowiedniej funkcji.
\begin{lstlisting}[caption=Funkcja testująca wpływ parametru $k$ na przebieg wartości $d$.]
function alg1fun1(u, kstart, kstep, kstop)
    color = char('y', 'k', 'b', 'g', 'r', 'm');
    hold all;
    k = kstart;
    c = 1;
    
    while(k <= kstop)
        epsilon = 0.1;
        delta = 0.1
        u1(1) = u(1);
        u2(1) = u(2);
        y(1) = funkcja1([u(1) u(2)]);
        i = 1;
        d(1) = 1;
        
        while(epsilon < abs(d(i)))
            a=(funkcja1([u1(i)+delta u2(i)+delta]) - funkcja1([u1(i) u2(i)]));
            d(i+1) =  a / delta;
            u1(i+1) = u1(i) - k * d(i+1);
            u2(i+1) = u2(i) - k * d(i+1);
            y(i+1)= funkcja1([u1(i) u2(i)]);
            i = i + 1;
        end

        figure(1);
        hold on;
        plot(u1, y, '*b');  plot(u2, y, '*r');
        figure(2);
        hold on;
        plot(d, strcat('-', color(mod(c,6)+1)));
        
        c = c + 1;
        k = k + kstep;
        clear u1; clear u2; clear d; clear y;
    end
end
\end{lstlisting}

\subsubsection{Obiekt (2)}
%--------------------------------------------------------------------------------------------------------------------------------
%ZADANIE 2
%--------------------------------------------------------------------------------------------------------------------------------
\subsection{Symulacja systemu sterowania ekstremalnego – wersja 1. algorytmu.}

\subsubsection{Obiekt (1)}

\subsubsection{Obiekt (2)}
%--------------------------------------------------------------------------------------------------------------------------------
%ZADANIE 3
%--------------------------------------------------------------------------------------------------------------------------------
\subsection{Symulacja systemu sterowania ekstremalnego – wersja 1. algorytmu.}

\subsubsection{Obiekt (1)}

\subsubsection{Obiekt (2)}
\section{Wnioski.}\label{sec:wnioski}

\end{document}