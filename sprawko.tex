\documentclass[a4paper,10pt]{article}
\usepackage[utf8]{inputenc}
\usepackage{polski}

\title{Sprawozdanie ISS} 
\author{Adam Jordanek 168139, Tomasz Klimek 168092}
 
\begin{document}
\maketitle
 
 
\section{Wstęp}\label{sec:wstep}
%TODO skopiowane z listy (trzeba to przerobić)
Tworzenie opisów (modeli) matematycznych obiektów sterowania, a także wykorzystanie tych opisów do badania i analizy (modelowania) obiektów – są istotnymi czynnościami w trakcie projektowania informatycznych systemów sterowania. Do realizacji tych czynności w praktyce inżynierskiej powszechnie stosuje się narzędzie informatyczne Matlab wraz ze pecjalistycznym oprogramowaniem dodatkowym (tzw. toolbox’y) oraz nakładką Simulink.
 
\section{Zadanie 1 \textit{\small Tworzenie modeli matematycznych}}\label{sec:zad1}
\begin{itemize}
\item Inrecyjny 1-rzędu
\item Inercyjny 2-rzędu
\item Całkujący rzeczywisty
\item Różniczkujący rzeczywisty
\item Proporcionalny
\end{itemize}
\section{Zadanie 2 \textit{\small Wyznaczanie charakterystyk czasowych}}\label{sec:zad2}
\section{Zadanie 3 \textit{\small Wyznaczanie parametrów członów dynamicznych}}\label{sec:zad3}

 
\end{document}



