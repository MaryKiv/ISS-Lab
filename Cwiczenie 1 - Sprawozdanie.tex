\documentclass[a4paper,10pt]{article}
\usepackage[utf8]{inputenc}
\usepackage{polski}

\title{Sprawozdanie ISS\newline\small Ćwiczenie 1: Podstawy tworzenia opisów i modelowania obiektów sterowania}

\author{Adam Jordanek 168139, Tomasz Klimek 168092}

\begin{document}
\maketitle

\section{Wstęp}\label{sec:wstęp}
%TODO skopiowane z listy (trzeba to przerobić)
Tworzenie opisów (modeli) matematycznych obiektów sterowania, a także wykorzystanie tych opisów do badania i analizy (modelowania) obiektów – są istotnymi czynnościami w trakcie projektowania informatycznych systemów sterowania. Do realizacji tych czynności w praktyce inżynierskiej powszechnie stosuje się narzędzie informatyczne Matlab wraz ze specjalistycznym oprogramowaniem dodatkowym (tzw. toolbox’y) oraz nakładką Simulink.

\section{Zadanie 1 \textit{\small Tworzenie modeli matematycznych}}\label{sec:zad1}

\begin{itemize}
\item Inercyjny 1-rzędu

\begin{itemize}
\item math	
	\begin{eqnarray}
		G(s)= {k \over {Ts + 1}} = {Y(s) \over U(s)}
	\end{eqnarray}
\item Równanie różniczkowe
	\begin{eqnarray}
		Y(s) = {k \over {Ts + 1}} U(s)\\
		\nonumber TsY(s) + Y(s) = kU(s)\\
		\nonumber T\dot{y}(t) + y(t) = ku(t)
	\end{eqnarray}
\item Równanie stanu
\newline Ogólna postać równania stanu: %COPY
	\begin{eqnarray}
		\left\{
			\begin{array}{l}
				x(t) = Ax(t) + Bu(t)\\
				y(t) = C^{T}x(t)
			\end{array} \right.
	\end{eqnarray}
Ponieważ na wyjściu obiektu jest pochodna 1-go rzędu to wektor stanu x(t) będzie reprezentowany przez wektor jednoelementowy x(t)=y(t). Można wprowadzić więc zapis:
	\begin{eqnarray}
		\dot{x}(t) = ax(t) + bu(t)\\
		\nonumber y(t) = x(t)
	\end{eqnarray}
Przekształcenie z równania różniczkowego na równanie stanu: %COPY
	\begin{eqnarray}
		T\dot{y}(t) + y(t) = ku(t)\\
		\nonumber T\dot{x}(t) + x(t) = ku(t)\\
		\nonumber T\dot{x}(t) = ku(t) - x(t)\\
		\nonumber \dot{x}(t) = -{x \over T}(t) + {k \over T}u(t)
	\end{eqnarray}
\end{itemize}

\item Inercyjny 2-rzędu
\begin{itemize}
\item Transmitancja
	\begin{eqnarray}
		G(s)= {k \over \left({T_{1}s + 1}\right) \left({T_{2}s + 1}\right)} = {Y(s) \over U(s)}
	\end{eqnarray}
\item Równanie różniczkowe
	\begin{eqnarray}
		Y(s) = {k \over {\left({T_{1}s + 1}\right) \left({T_{2}s + 1}\right)}} U(s)\\
		\nonumber Y(s)\left({T_{1}s + 1}\right) \left({T_{2}s + 1}\right)= kU(s)\\
		\nonumber T_{1}T_{2}s^{2}Y(s) + \left({T_{1} + T_{2}}\right) s Y(s) + Y(s)= kU(s)\\
		\nonumber T_{1}T_{2}\ddot{y}(t) + \left({T_{1} + T_{2}}\right) \dot{y}(t) + y(t)= k u(t)
	\end{eqnarray}
\item Równanie stanu
%TODO
\end{itemize}

\item Całkujący rzeczywisty

\item Różniczkujący rzeczywisty
\begin{itemize}
\item Transmitancja
	\begin{eqnarray}
		G(s)= {ks \over \left({Ts + 1}\right)} = {Y(s) \over U(s)}
	\end{eqnarray}
\item Równanie różniczkowe
	\begin{eqnarray}
		Y(s) = {ks \over \left({Ts + 1}\right)}  U(s)\\
		\nonumber TsY(s)+Y(s)=k s U(s)\\
		\nonumber T\dot{y}(t)+y=k\dot{u}(t)
	\end{eqnarray}
\item Równanie stanu
%TODO
\end{itemize}

\item Proporcjonalny

\end{itemize}

\section{Zadanie 2 \textit{\small Wyznaczanie charakterystyk czasowych}}\label{sec:zad2}
\section{Zadanie 3 \textit{\small Wyznaczanie parametrów członów dynamicznych}}\label{sec:zad3}

\end{document}