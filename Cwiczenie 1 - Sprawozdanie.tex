\documentclass[a4paper,10pt]{article}
\usepackage[utf8]{inputenc}
\usepackage{polski}
\usepackage{graphicx}
\usepackage{listings}
\usepackage[usenames,dvipsnames]{color}


\title{Sprawozdanie ISS\newline\small Ćwiczenie 1: Podstawy tworzenia opisów i modelowania obiektów sterowania}

\author{Adam Jordanek 168139, Tomasz Klimek 168092}

\lstset{
    language=Matlab,
    basicstyle=\scriptsize,
    aboveskip={1.5\baselineskip},
    columns=fixed,
    showstringspaces=false,
    extendedchars=true,
    breaklines=true,
    tabsize=4,
    prebreak = \raisebox{0ex}[0ex][0ex]{\ensuremath{\hookleftarrow}},
    frame=single,
    showtabs=false,
    showspaces=false,
    showstringspaces=false,
    identifierstyle=\ttfamily,
    keywordstyle=\color[rgb]{0,0,1},
    commentstyle=\color[rgb]{0.133,0.545,0.133},
    stringstyle=\color[rgb]{0.627,0.126,0.941},
    numbers=left,
    numberstyle=\tiny,
    stepnumber=1,
    numbersep=5pt,
    captionpos=b,
    escapeinside={\%*}{*)}
}



\begin{document}
\maketitle

\section{Wstęp}\label{sec:wstęp}
%TODO skopiowane z listy (trzeba to przerobić)
Tworzenie opisów (modeli) matematycznych obiektów sterowania, a także wykorzystanie tych opisów do badania i analizy (modelowania) obiektów – są istotnymi czynnościami w trakcie projektowania informatycznych systemów sterowania. Do realizacji tych czynności w praktyce inżynierskiej powszechnie stosuje się narzędzie informatyczne Matlab wraz ze specjalistycznym oprogramowaniem dodatkowym (tzw. toolbox’y) oraz nakładką Simulink.

\section{Zadanie 1 \textit{\small Tworzenie modeli matematycznych}}\label{sec:zad1}

\begin{itemize}
\item Inercyjny 1-rzędu

\begin{itemize}
\item Transmitancja	
	\begin{eqnarray}
		G(s)= {k \over {Ts + 1}} = {Y(s) \over U(s)}
	\end{eqnarray}
\item Równanie różniczkowe
	\begin{eqnarray}
		\nonumber Y(s) = {k \over {Ts + 1}} U(s)\\ 
		\nonumber TsY(s) + Y(s) = kU(s)\\
		T\dot{y}(t) + y(t) = ku(t) \label{eq:iter_rozn}
	\end{eqnarray}
\item Równanie stanu
\newline Ogólna postać równania stanu:
	\begin{eqnarray}
		\nonumber \left\{ \begin{array}{l}
			\dot{x}(t) = Ax(t) + Bu(t)\\
			y(t) = C^{T}x(t)
		\end{array} \right.
	\end{eqnarray}
W równaniu różniczkowym wyjście obiektu jest najwyżej w pierwszej pochodnej, więc mamy do czynienia z obiektem stopnia pierwszego, a wektor stanu x(t) będzie reprezentowany przez jednoelementowy wektor $x(t) = y(t)$, parametry A, B i C będą zwykłymi parametrami liczbowymi, a równanie stanu można zapisać jako:
	\begin{eqnarray}
		\nonumber \left\{ \begin{array}{l}
			\dot{x}(t) = ax(t) + bu(t)\\
			y(t) = cx(t)
		\end{array} \right.
	\end{eqnarray}
Aby uzyskać równanie stanu wystarczy zauważyć, że c=1, oraz przekształcić równanie \ref{eq:iter_rozn} w następujący sposób:
	\begin{eqnarray}
		\nonumber T\dot{y}(t) + y(t) = ku(t)\\
		\nonumber T\dot{x}(t) + x(t) = ku(t)\\
		\nonumber T\dot{x}(t) = - x(t) + u(t)\\
		\dot{x}(t) = -{1 \over T}x(t) + {k \over T}u(t)
	\end{eqnarray}
W ten sposób przyrównując otrzymaliśmy gotowe równanie stanu.
	\begin{eqnarray}
		\left\{ \begin{array}{l}
			\dot{x}(t) = -{1 \over T}x(t) + {k \over T}u(t)\\
			y(t) = x(t)
		\end{array} \right.
	\end{eqnarray}
\item Opis w Matlabie
\newline Opis za pomocą równania stanu w Matlabie uzyskuje się przez podanie współczynników wielomianów $Y(s)$ oraz $U(s)$ z transmitancji (zaczynając od najwyższej) do funkcji $tf2ss(Y, U)$, która zwróci parametry równania stanu.
Ponieważ w Matlabie nie możemy stosować stałych jak k, czy T, podamy w ich miejsce wartości liczbowe.
\newline\newline Dla $k=1$, $T=2$
\newline $>>[A,B,C] = tf2ss([1], [2 \ 1])$
\newline Zwraca wynik:
$A = -0.5, B = 1, C = 0.5000$

\end{itemize}

\item Inercyjny 2-rzędu
\begin{itemize}
\item Transmitancja
	\begin{eqnarray}
		G(s)= {k \over \left({T_{1}s + 1}\right) \left({T_{2}s + 1}\right)} = {Y(s) \over U(s)}
	\end{eqnarray}
\item Równanie różniczkowe
	\begin{eqnarray}
		\nonumber Y(s) = {k \over {\left({T_{1}s + 1}\right) \left({T_{2}s + 1}\right)}} U(s)\\
		\nonumber Y(s)\left({T_{1}s + 1}\right) \left({T_{2}s + 1}\right)= kU(s)\\
		\nonumber T_{1}T_{2}s^{2}Y(s) + \left({T_{1} + T_{2}}\right) s Y(s) + Y(s)= kU(s)\\
		T_{1}T_{2}\ddot{y}(t) + \left({T_{1} + T_{2}}\right) \dot{y}(t) + y(t)= k u(t)
	\end{eqnarray}
\item Równanie stanu
%TODO
\end{itemize}

\item Całkujący rzeczywisty

\begin{itemize}
\item Transmitancja	
	\begin{eqnarray}
		G(s)= {k \over {s \left( Ts + 1 \right) }} = {Y(s) \over U(s)}
	\end{eqnarray}
\item Równanie różniczkowe
	\begin{eqnarray} 
		\nonumber Y(s) = {k \over {s \left( Ts + 1 \right) }} U(s)\\ \label{eq:calk_rozn}
		\nonumber Ts^{2}Y(s) + sY(s) = kU(s)\\
		T\ddot{y}(t) + \dot{y}(t) = ku(t)
	\end{eqnarray}
\item Równanie stanu
\newline W równaniu różniczkowym wyjście obiektu jest najwyżej w drugiej pochodnej, więc mamy do czynienia z obiektem stopnia drugiego, a wektor stanu x(t) będzie reprezentowany przez dwuelementowy wektor:
 	\begin{eqnarray}
		\nonumber x(t) = \left[ 
			\begin{array}{l}
				x_{1}(t)\\
				x_{2}(t)
			\end{array}
		\right] = \left[ 
			\begin{array}{l}
				y(t)\\
				\dot{y}(t)
			\end{array}
		\right]
	\end{eqnarray}
Parametry A, B i C będą natomiast miały postać:
	\begin{eqnarray}
		\nonumber A = \left[ 
			\begin{array}{ll}
				a_{11} & a_{12}\\
				a_{21} & a_{22}
			\end{array}
		\right], B = \left[ 
			\begin{array}{l}
				b_{1}\\
				b_{2}
			\end{array}
		\right], C = \left[ 
			\begin{array}{l}
				c_{1}\\
				c_{2}
			\end{array}
		\right]
	\end{eqnarray}
Aby uzyskać równanie stanu przekształcamy równanie \ref{eq:calk_rozn} otrzynując:
	\begin{eqnarray}
		\nonumber T\ddot{y}(t) + \dot{y}(t) = ku(t)\\
		\nonumber T\dot{x}(t) + x(t) = ku(t)\\
		\nonumber T\dot{x}(t) = ku(t) - x(t)\\
		\dot{x}(t) = -{x \over T}(t) + {k \over T}u(t)
	\end{eqnarray}
Z otrzymanego równania możemy wywnioskować wartości parametrów A, B i C, oraz zapisać ostateczną postać równania stanu:
	\begin{eqnarray}
		\left\{
			\begin{array}{l}
				x(t) = \left[ 
			\begin{array}{ll}
				0 & 1\\
				0 & -{1 \over T}
			\end{array}
		\right] x(t) + \left[ 
			\begin{array}{l}
				0\\
				{k \over T}
			\end{array}
		\right] u(t)\\
				y(t) = \left[ 
			\begin{array}{ll}
				1 & 0
			\end{array}
		\right]x(t)
			\end{array} \right.
	\end{eqnarray}
\end{itemize}
\item Opis w Matlabie
\newline Opis uzyskamy analogicznie jak w poprzednich przykładach.
\newline\newline Dla $k=1$, $T=2$
\newline $>>[A,B,C] = tf2ss([1], [2 \ 1 \ 0])$
\newline Zwraca wynik:
$A = \left[ \begin{array}{ll} -0.5 & 0\\ 1 & 0 \end{array} \right], B = \left[ \begin{array}{l} 1\\ 0 \end{array} \right], C = \left[ \begin{array}{ll} 0 & 0.5 \end{array} \right]$

\item Różniczkujący rzeczywisty
\begin{itemize}
\item Transmitancja
	\begin{eqnarray}
		G(s)= {ks \over \left({Ts + 1}\right)} = {Y(s) \over U(s)}
	\end{eqnarray}
\item Równanie różniczkowe
	\begin{eqnarray}
		Y(s) = {ks \over \left({Ts + 1}\right)}  U(s)\\
		\nonumber TsY(s)+Y(s)=k s U(s)\\
		\nonumber T\dot{y}(t)+y=k\dot{u}(t)
	\end{eqnarray}
\item Równanie stanu
%TODO
\end{itemize}

\item Proporcjonalny
\begin{itemize}
\item Transmitancja	
	\begin{eqnarray}
		G(s)= {k}= {Y(s) \over U(s)}
	\end{eqnarray}
\item Równanie różniczkowe
	\begin{eqnarray} 
		\nonumber Y(s) = k U(s)\\ \label{eq:prop_rozn}
		y(t)= ku(t)
	\end{eqnarray}
\item Równanie stanu
	Nie ma opisu w przestrzeni stanów.
\end{itemize}

\end{itemize}

\section{Zadanie 2 \textit{\small Wyznaczanie charakterystyk czasowych}}\label{sec:zad2}
\begin{enumerate}
\item A. Uzyskanie charakterystyki impulsowej i charakterystyki skokowej w formie wykresu
 przy wykorzystaniu odpowiednich funkcji MathLaba.\\
\item B. Napisanie procedury w Matlabie, umożliwiającej wykreślenie charakterystyki
impulsowej i charakterystyki czasowej na podstawie opisu w przestrzeni stanu.\\
	\begin{eqnarray} 
		\dot{x}(t=n \Delta t ) = {x_{n+1}-x_{n} \over {\Delta t}}
	\end{eqnarray}
	\begin{eqnarray} 
		x_{n}=x(t=n \Delta t)
	\end{eqnarray}
	\begin{eqnarray} 
		u_{n}=u(t=n \Delta t)
	\end{eqnarray}
	\begin{eqnarray} 
		y_{n}=y(t=n \Delta t)
	\end{eqnarray}\\
	Dodatkowo korzystamy z:\\
	\begin{eqnarray} 
		\nonumber \dot{x}_{n}=Ax_{n}+Bu_{n}
		\nonumber y_{n}=Cx_{n}
	\end{eqnarray}\\
	Otrzymujemy układ dyskretny:\\
	\begin{eqnarray} 
		\nonumber {x}_{n+1}=(Ax_{n}+Bu_{n}) \Delta t +x_{n}
		\nonumber y_{n}=Cx_{n}
	\end{eqnarray}\\
	Taki układ można już zastosować do napisania funkcji:
\begin{lstlisting}[caption=funkcja mImpulse.m]
function p=mImpulse( A,B,C, dt, n)
	y = zeros(1,n);
	x = zeros(size(A),n);
	x(1,1)=1;
	t = 0:dt:dt*(n-1);
	u=1;
	for i=1:n
		y(1,i) = C*x(:,i);
		x(:,i+1) = (A*x(:,i)+B*u)*dt + x(:,i);
		u = 0;
	end
	p=plot(t,y(1,:))
end
\end{lstlisting}
Funkcja zwraca wykres charakterystyki impulsowej. Na wejście dostaje parametry opisu A,B,C,
wartość dyskretnego przejścia dt oraz ilość kroków (przedział wykresu [0, dt*(n-1)]). Po pierwszej
iteracji wartość u jest zmieniana z 1 na 0 co odpowiada impulsowi. %COPY

\begin{lstlisting}[caption=funkcja mStep.m]
function p=mStep( A,B,C, dt, n)
	y = zeros(1,n);
	x = zeros(size(A),n);
	t = 0:dt:dt*(n-1);
	u=1;
	for i=1:n
		y(1,i) = C*x(:,i);
		x(:,i+1) = (A*x(:,i)+B*u)*dt + x(:,i);
	end
	p=plot(t,y(1,:))
end
\end{lstlisting}
Funkcja zwraca wykres charakterystyki skokowej. Na wejście dostaje parametry opisu A,B,C, wartość
dyskretnego przejścia dt oraz ilość kroków (przedział wykresu [0, dt*(n-1)]). Wartość u stale jest
równa 1 co odpowiada sygnałowi skoku.%COPY

	

\item C. Narysowanie przebiegu obu charakterystyk w czasie, z wykorzystaniem procedur
napisanych w p. B\\
\item D. Porównanie charakterystyk wyznaczonych w p. C z odpowiednimi charakterystykami
uzyskanymi w p. A.\\

\end{enumerate}
\section{Zadanie 3 \textit{\small Wyznaczanie parametrów członów dynamicznych}}\label{sec:zad3}

Przepisane dla inercyjnego:
	\begin{eqnarray} 
		g(t)={k \over T} e^{-{t \over T}}
	\end{eqnarray}\\
	
Aby odczytać wartość k i T postępujemy w następujący sposób:
Wyliczamy dla punktu t=0:

	\begin{eqnarray} 
		g(0)={k \over T}
	\end{eqnarray}\\
	Liczymy pochodną:
	\begin{eqnarray} 
		\dot{g}(t)=-{k \over T^{2}} e^{-{t \over T}}
	\end{eqnarray}\\
	Obliczamy wartość pochodnej w punkcjie t=0
	\begin{eqnarray} 
		\dot{g}(0)=-{k \over T^{2}}=-{1 \over T^{2}}g(0)
	\end{eqnarray}\\
Przedstawiamy pochodną dla t=0 jako współczynnik kierunkowy stycznej (s(t)). Załóżmy, że dla t’
dana styczna osiąga wartość 0 (przecina oś Ot). Wartość tego współczynnika ma wartość (wiemy
także, że s(0)=g(0)):



\end{document}